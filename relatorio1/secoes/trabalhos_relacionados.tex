\section{Trabalhos Relacionados}
Existem produtos que possuem algumas similaridades com a proposta abordada por este trabalho. Tais produtos servem como inspiração para que se desenvolva um produto com uma melhor qualidade. Como exemplo de tais produtos pode-se citar:
\begin{itemize}
\item Fresadora CNC para a usinagem em 5 eixos: Esta fresadora consiste em um conjunto mecânico, módulos de hardware e um software de interface gráfica. A estrutura mecânica da máquina permite a movimentação linear dos eixos X, Y e Z, operando juntamente com a rotação da mesa de trabalho nos eixos B e C, totalizando cinco eixos. Tal máquina é utilizada em usinagem de peças de pequenas dimensões (WERNER, 2015).
\item Máquina CNC de 3 eixos utilizando tecnologias livres: Máquina-ferramenta de Comando Numérico Computadorizado (CNC) de três eixos lineares, baseada em tecnologias livres, de baixo custo e didática, para a utilização em universidades, escolas e demais interessados na tecnologia CNC. O projeto desenvolvido teve um custo aproximado de dois mil reais (KRUG e SILVA, 2008).
\item Dispositivo de Fresamento Controlado por CNC: Máquina-ferramenta de Comando Numérico Computadorizado para fins acadêmicos. A mesma é utilizada em usinagem de peças de alumínio, pois o mesmo possui alta resistência mecânica, média usinabilidade e leveza, apesar de possuir um custo mais elevado (COUTINHO e SANTIAGO, 2014).
\item Equipamento mecânico com controle numérico computadorizado para produção de protótipos em escala: Equipamento mecânico controlado numericamente por computador (CNC) para produção de protótipos em escala, principalmente para fins didáticos. Tal equipamento mecânico poderá ser utilizado tanto no aprendizado em laboratório de sistemas CNC, quanto na criação de protótipos em escala para as áreas de Desenho Mecânico, Elementos de Máquinas e Projetos Mecânicos (SILVEIRA, 2007).
\item Desenvolvimento de uma Máquina Fresadora CNC Didática: trabalho é parte do desenvolvimento de uma Célula Flexível de Manufatura que já dispõe de um torno CNC didático. O mesmo propõe-se a concepção, projeto, construção e validação de uma máquina fresadora CNC didática. A concepção é baseada nas fresadoras CNC oferecidas pelo mercado nacional (LYRA, 2010).
\item Projeto de Fresadora CNC com Plataforma Livre Arduino: Protótipo de fresadora CNC de baixo custo com um programa de livre acesso que facilita o processo de usinagem. Utilizou-se o firmware GrBl ao Arduino, o mesmo trata-se de um interpretador de comandos de código aberto mais utilizado em projetos DIY (do it yourself), o qual se comunica com o Gstudio. Onde este permite a criação de desenhos, alteração do código G, e visualização em tempo real do funcionamento da máquina (FACHIM, 2013). 
\end{itemize}    

