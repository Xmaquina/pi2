\section{Introdução}
\subsection{Contextualização}
    Desde muito tempo vimos que as máquinas vêm sendo utilizadas na fabricação de peças, 
fazendo esse processo mais rápido e preciso. Com as máquinas CNC (Computer Numeric Control) 
e o marcador a laser não são diferentes, elas aceleram o processo e permitem uma precisão muito maior que a do homem. 
Além da precisão e rapidez, essas máquinas podem otimizar o desperdício de material fazendo que a empresa economize uma 
parte do seu capital de investimento.

A construção de uma 2 em 1, uma CNC e um marcador a laser traria uma certa economia e praticidade para seu usuário. Haja vista que com poucos ajustes podemos inverter o modo de operação de nossa máquina de CNC para marcador a laser e vice-versa. E com a melhoria que a parte de software vai dar na máquina fazendo com que o manuseio da mesma seja praticamente todo pela rede, não teremos mais a necessidade de utilizar cabos e componentes externos a máquina para a sua comunicação, eliminando problemas externos de uma máquina deste modelo.

\subsection{Objetivo Geral}
Construção de uma máquina com as funções de fresadora CNC e marcadora a laser, funcionando 
conjuntamente com um software que envia informações para a máquina sem o auxílio de cabos.



